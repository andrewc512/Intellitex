\documentclass[11pt]{article}
\usepackage[margin=1in]{geometry}
\usepackage[T1]{fontenc}
\usepackage[utf8]{inputenc}
\usepackage{hyperref}
\usepackage{parskip}
\usepackage{enumitem}
\usepackage{listings}
\usepackage{xcolor}

\definecolor{codebg}{RGB}{248,248,248}
\lstset{
  basicstyle=\ttfamily\small,
  backgroundcolor=\color{codebg},
  frame=single,
  framerule=0pt,
  xleftmargin=0.5em,
  xrightmargin=0.5em,
  aboveskip=0.75em,
  belowskip=0.75em
}

\title{.itek Language Reference (Transpiler-Specified)}
\author{IntelliTex}
\date{February 27, 2026}

\begin{document}
\maketitle

\section*{Scope}
This document describes the \texttt{.itek} resume language \emph{as implemented in the IntelliTex transpiler and parser}. It reflects actual parsing and rendering behavior, not just the intended syntax.

\section*{High-Level Model}
\begin{itemize}[leftmargin=1.25em]
  \item The file is line-oriented; each non-empty line is parsed independently.
  \item Indentation is ignored (each line is trimmed before parsing).
  \item A document contains a name and a list of sections in the order they appear.
  \item Sections contain fields, optional entries, and optional bullets.
  \item Entries contain fields and bullet points.
\end{itemize}

\section*{Grammar (Line-Oriented)}
\begin{lstlisting}
@resume <Full Name>          document declaration

#<section>                   section header (lowercased by parser)

  <key>: <value>             field with plain value
  <key>: "<value>"           field with quoted value
  <key>: <url>               field with URL or email in angle brackets

  company <Name>             entry in #experience or #leadership
  project <Name>             entry in #projects
  organization <Name>        entry in #leadership
  ## <Name>                  generic entry (works in any section)

    <key>: <value>           field on an entry
    * <text>                 bullet point on an entry

the parser also allows:
  company: <Name>            with a colon after the marker
  project: "<Name>"          quoted entry name
  organization: <Name>
\end{lstlisting}

\textbf{Notes:}
\begin{itemize}[leftmargin=1.25em]
  \item The parser does not enforce that \texttt{@resume} is the first line, but the app expects it to be.
  \item Multiple \texttt{@resume} lines will overwrite the name; the last one wins.
  \item Section names and field keys are lowercased by the parser.
  \item Blank lines are ignored.
\end{itemize}

\section*{Built-In Sections and Rendering}
\subsection*{\#socials (header-only)}
Fields read from this section are rendered into the header line under the name. This section itself is \emph{not} rendered as a standalone section.

Recognized fields (all optional):
\begin{itemize}[leftmargin=1.25em]
  \item \texttt{number} (phone)\;\;--\;\;formatted if 10 digits.
  \item \texttt{email}
  \item \texttt{linkedin}
  \item \texttt{github}
  \item \texttt{website}
  \item \texttt{portfolio}
\end{itemize}

If a field uses \texttt{<...>}, the value is treated as a URL (or email) and displayed without the protocol in the header.

\subsection*{\#education}
Rendered as \texttt{Education}. Each entry is rendered as a \texttt{\textbackslash resumeSubheading}.

Fields used (entry or section-level):
\begin{itemize}[leftmargin=1.25em]
  \item \texttt{school} (if no entries exist, this is used as the entry name)
  \item \texttt{loc} or \texttt{location}
  \item \texttt{degree}
  \item \texttt{gpa} (appended to degree if present)
  \item \texttt{grad} or \texttt{graduation}
  \item \texttt{courses} or \texttt{coursework} (rendered as a bullet under the entry)
\end{itemize}

\subsection*{\#experience and \#leadership}
Rendered using the same layout. Each entry uses:
\begin{itemize}[leftmargin=1.25em]
  \item Entry name (company/organization/\#\# name)
  \item \texttt{loc} or \texttt{location}
  \item \texttt{role} or \texttt{title}
  \item \texttt{date}
  \item Bullet points (\texttt{* ...})
\end{itemize}

\subsection*{\#projects}
Each entry uses:
\begin{itemize}[leftmargin=1.25em]
  \item Entry name
  \item \texttt{stack} or \texttt{technologies} or \texttt{tech}
  \item \texttt{date}
  \item Bullet points
\end{itemize}

\subsection*{\#skills}
Rendered as \texttt{Technical Skills}. All key/value pairs in the section are emitted. Keys are capitalized and each item is printed on its own line.

\subsection*{Unknown Sections}
If a section type is not recognized but it has entries, it is rendered using the experience-style layout with the section title capitalized. If it has no entries, it is omitted from output.

\section*{Parsing Rules and Edge Cases}
\begin{itemize}[leftmargin=1.25em]
  \item \textbf{Entry markers:} \texttt{company}, \texttt{project}, \texttt{organization} can be used with a space or a colon (e.g. \texttt{company Foo} or \texttt{company: Foo}).
  \item \textbf{Generic entries:} \texttt{\#\# <Name>} is valid in any section and does not imply any semantics beyond entry creation.
  \item \textbf{Bullets:} Bullets attach to the most recent entry. If no entry exists, bullets attach to the section, but section-level bullets are not rendered by the current transpiler.
  \item \textbf{URL fields:} Values wrapped in \texttt{<...>} are stored as \texttt{\{text, url\}}. In most sections this behaves like a plain string; in \texttt{\#socials} it is used to create links.
  \item \textbf{Escaping:} The transpiler escapes LaTeX special characters (\texttt{\textbackslash}, \texttt{\&}, \texttt{\%}, \texttt{\$}, \texttt{\#}, \texttt{\_}, \texttt{\{}, \texttt{\}}, \texttt{\~}, \texttt{\^}).
\end{itemize}

\section*{Minimal Example}
\begin{lstlisting}
@resume Jane Doe

#socials
  email: <jane@example.com>
  github: <https://github.com/janedoe>

#education
  school: "State University"
  degree: "B.S. Computer Science"
  grad: "May 2026"

#experience
  company Example Corp
    role: "Software Engineer"
    loc: "Remote"
    date: "Jun 2024 - Present"
    * Built internal tools that reduced manual work by 40%
\end{lstlisting}

\section*{Transpilation Output Summary}
The transpiler emits a LaTeX resume using a fixed template (Jake's-Resume style). It always includes:
\begin{itemize}[leftmargin=1.25em]
  \item A preamble with common resume packages (\texttt{fullpage}, \texttt{titlesec}, \texttt{enumitem}, \texttt{hyperref}, etc.).
  \item A centered header with the name and socials line.
  \item Rendered sections in the order they appear in the source (excluding \texttt{\#socials}).
\end{itemize}

\section*{Source of Truth}
All behavior in this document is derived from:
\begin{itemize}[leftmargin=1.25em]
  \item \texttt{electron/itek/parser.js}
  \item \texttt{electron/itek/transpiler.js}
  \item \texttt{examples/jayden-tan.itek}
  \item \texttt{electron/agent/prompts.js}
\end{itemize}

\end{document}
